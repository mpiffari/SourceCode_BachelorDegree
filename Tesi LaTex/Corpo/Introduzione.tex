%======================= DOCUMENTO ======================================
Questo progetto si inserisce all'interno del percorso di ricerca avviato dal laboratorio del CAL \emph{(Control systems and Automation Laboratory)} dell'Università degli studi di Bergamo, il cui obbiettivo è stato quello di creare e gestire un sistema autonomo di pick and place da inserire in ambiente agricolo.

Il fulcro dello studio condotto è stato quello di eseguire uno start up del sistema: in un primo momento si è andato a studiare la componentistica hardware, composta dal manipolatore \emph{ABB IRB 120} e dal controllore \emph{IRC5 Compact}, mettendo in atto un lavoro di \emph{unboxing} del sistema stesso, installando i collegamenti fisici tra il manipolatore e il controllore e cablando la connessione alla rete elettrica.

Dopo una prima fase appunto più pratica, che ha portato sostanzialmente alla cablatura del sistema unitamente ad un primo approccio al mondo della robotica (il cui report è contenuto nella parte~\vref{part:Il_sistema}), il focus del progetto si è spostato sulla parte di controllo e gestione della traiettoria del manipolatore: sono stati esaminati soprattutto la parte di comunicazione PC/Controller e la cinematica del braccio robotico, implementata poi, in via sperimentale e non definitiva, tramite software. Questa è proprio una peculiarità del mondo della robotica in cui si ha la necessità di focalizzarsi contemporaneamente sia sul design del robot (e quindi sulla geometria e sulle matrici caratterizzanti il movimento) sia sulla tipologia di utilizzo dello stesso robot da parte dell'utente finale. 

Saranno quindi qui riportate le analisi condotte sul sistema a livello hardware e lo studio del controllo dal punto di vista software, in una forma più manualistica, per poter aiutare eventuali progetti futuri riguardanti lo sviluppo di sistemi basati su questa famiglia di manipolatori \emph{ABB}, per mezzo dello studio della componentistica nella parte(parte~\vref{part:Il_sistema}) unitamente alle possibilità di controllo analizzate invece nella parte~\vref{part:Il_controllo}.

L'obbiettivo del progetto, chiudendo così la parte introduttiva, è quindi quello di andare a capire il funzionamento del manipolatore, studiandone la componente software \emph{ABB}, prodotta in RAPID: una volta consolidata la conoscenza del dominio applicativo "software" siamo andati poi a validare il modo migliore per poter governare il movimento del manipolatore stesso tramite PC, analizzando tutte le possibilità di comunicazione disponibili per poter fare dialogare il controllore \emph{IRC5 Compact} con il braccio \emph{ABB IRB 120}.
%========================================================================